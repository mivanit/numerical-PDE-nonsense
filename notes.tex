% Options for packages loaded elsewhere
\PassOptionsToPackage{unicode}{hyperref}
\PassOptionsToPackage{hyphens}{url}
%
\documentclass[
]{article}
\author{}
\date{}

\usepackage{amsmath,amssymb}
\usepackage{lmodern}
\usepackage{iftex}
\ifPDFTeX
  \usepackage[T1]{fontenc}
  \usepackage[utf8]{inputenc}
  \usepackage{textcomp} % provide euro and other symbols
\else % if luatex or xetex
  \usepackage{unicode-math}
  \defaultfontfeatures{Scale=MatchLowercase}
  \defaultfontfeatures[\rmfamily]{Ligatures=TeX,Scale=1}
\fi
% Use upquote if available, for straight quotes in verbatim environments
\IfFileExists{upquote.sty}{\usepackage{upquote}}{}
\IfFileExists{microtype.sty}{% use microtype if available
  \usepackage[]{microtype}
  \UseMicrotypeSet[protrusion]{basicmath} % disable protrusion for tt fonts
}{}
\makeatletter
\@ifundefined{KOMAClassName}{% if non-KOMA class
  \IfFileExists{parskip.sty}{%
    \usepackage{parskip}
  }{% else
    \setlength{\parindent}{0pt}
    \setlength{\parskip}{6pt plus 2pt minus 1pt}}
}{% if KOMA class
  \KOMAoptions{parskip=half}}
\makeatother
\usepackage{xcolor}
\IfFileExists{xurl.sty}{\usepackage{xurl}}{} % add URL line breaks if available
\IfFileExists{bookmark.sty}{\usepackage{bookmark}}{\usepackage{hyperref}}
\hypersetup{
  hidelinks,
  pdfcreator={LaTeX via pandoc}}
\urlstyle{same} % disable monospaced font for URLs
\setlength{\emergencystretch}{3em} % prevent overfull lines
\providecommand{\tightlist}{%
  \setlength{\itemsep}{0pt}\setlength{\parskip}{0pt}}
\setcounter{secnumdepth}{-\maxdimen} % remove section numbering
\ifLuaTeX
  \usepackage{selnolig}  % disable illegal ligatures
\fi
\usepackage[]{biblatex}

\begin{document}

\hypertarget{lab-things-to-know-2021-12-13-1416}{%
\section{Lab Things to know (2021-12-13
14:16):}\label{lab-things-to-know-2021-12-13-1416}}

\hypertarget{lab-1}{%
\subsection{Lab 1}\label{lab-1}}

\begin{itemize}
\tightlist
\item
  Norms
\item
  order of accuracy: \[
        E_{order} = \frac{
            \log \left( \frac{e_{old}}{ e_{new} } \right)
        }{
            \log \left( \frac{N_{new}}{N_{old}} \right)
        }
    \]
\item
  Fourier series
\end{itemize}

\hypertarget{lab-2}{%
\subsection{Lab 2}\label{lab-2}}

\begin{itemize}
\tightlist
\item
  Fourier series
\item
  assumptions required for Fourier expansion

  \begin{itemize}
  \tightlist
  \item
    \(C_1\) piecewise

    \begin{itemize}
    \tightlist
    \item
      continuous in the first derivative, piecewise
    \item
      finitely many discontinuities over the interval
    \end{itemize}
  \item
    \(L_2\) (square integrable)

    \begin{itemize}
    \tightlist
    \item
      if \(C_1\) piecewise, that is a stricted condition than \(L_2\)
    \end{itemize}
  \item
    periodic
  \end{itemize}
\item
  Bessel's equality

  \begin{itemize}
  \tightlist
  \item
    sum of the square of the Fourier coefficients is the norm
  \end{itemize}
\item
  \(\ell_2\) error - \(\Vert\cdot\Vert_2\) of the difference
\end{itemize}

\hypertarget{lab-3}{%
\subsection{Lab 3}\label{lab-3}}

\begin{itemize}
\tightlist
\item
  difference operators \(D_0, D_-, D_+\)
\item
  general central divided differences formula \[
        D_0^k f(x) = (2h)^{-k} 
        \sum_{j \in \mathbb{N}_k }
        (-1)^j \binom{k}{j}
        f( x + (k - 2j) h)
        \qquad
        \forall \ k \in \mathbb{Z}_{\geq 0} 
    \]
\end{itemize}

\hypertarget{lab-4}{%
\subsection{Lab 4}\label{lab-4}}

\begin{itemize}
\tightlist
\item
  CFL

  \begin{itemize}
  \tightlist
  \item
    CFL is the numerical wavespeed
  \end{itemize}
\item
  amplification factor \(Q\)

  \begin{itemize}
  \tightlist
  \item
    plug Fourier modes into scheme
  \item
    \(Q\) has the CFL baked into it
  \end{itemize}
\item
  determining bounds on the CFL

  \begin{itemize}
  \tightlist
  \item
    \(|Q| \leq 1\) gives condition on CFL
  \item
    if \(|Q| = 1\), then conserving energy
  \end{itemize}
\item
  energy analysis

  \begin{itemize}
  \tightlist
  \item
    \textbf{probably will be on exam}
  \item
    usually defined with \(2\)-norm
  \item
    take the derivative, work through until you get the boundary terms
  \item
    put inside the governing equation??
  \item
    energy conserving means derivative of energy will be 0
  \end{itemize}
\end{itemize}

\begin{quote}
energy analysis example for \(u_t + a u_x = 0\)
\[ u \cdot u_t = - a \cdot u_x \cdot u \]
\[ \frac{d u^2}{dt} = 2 u \cdot u_t \]
\[ \frac{1}{2} \int_\Omega \frac{d|u|^2}{dt} d x \]
\end{quote}

\hypertarget{lab-5}{%
\subsection{Lab 5}\label{lab-5}}

\begin{itemize}
\tightlist
\item
  dissapative, dispersive, diffusive errors

  \begin{itemize}
  \tightlist
  \item
    use modified equation

    \begin{itemize}
    \tightlist
    \item
      take scheme and taylor expand everything
    \item
      take governing equation and shove it to the left, everything else
      on the right
    \item
      what's left on the right is truncation error

      \begin{itemize}
      \tightlist
      \item
        truncation error is before you take \(\Delta_x\), \(\Delta_t\)
        to \(\lambda\)
      \end{itemize}
    \item
      to get to the modified equation, put all the \(\Delta_t\) in terms
      of \(\lambda\) using the CFL definition (depends on eqn)
    \item
      then, use the governing equation to replace all time derivs with
      spatial ones (everything in terms of \(\Delta_x\)) \textbf{this is
      the modified equation}
    \end{itemize}
  \end{itemize}
\item
  look at leading error term in terms of \(\Delta_x\), look at order of
  spatial derivatives

  \begin{itemize}
  \tightlist
  \item
    if you have even derivatives, error is dissapative (melting)
  \item
    odd derivatives, dispersive (waves shifting)
  \item
    both is diffusive
  \end{itemize}
\item
  she may ask us to look at a plot and say things about the plot

  \begin{itemize}
  \tightlist
  \item
    for example, what is the error?
  \item
    shifts means dispersive
  \item
    decays means dissapative
  \item
    both means diffusive
  \end{itemize}
\item
  taking a second order and splitting it up into a system
\item
  decoupling a system

  \begin{itemize}
  \tightlist
  \item
    looking for the eigen modes?
  \item
    define 2 variables: \(u\) which you solve for, and \(v\) defined by
    \(v_t = u_x\)
  \item
    \textcite{joel} upload notes
  \end{itemize}
\end{itemize}

\hypertarget{lab-6}{%
\subsection{Lab 6}\label{lab-6}}

\begin{itemize}
\tightlist
\item
  implicit schemes

  \begin{itemize}
  \tightlist
  \item
    make matrix operators, take the inverse \[
    A u^{n+1} = B u^n
    \qquad\implies\qquad
    u^{n+1} = A^{-1} B u^n     
    \]
  \end{itemize}
\item
  amplification factor (matrix) for multiple timestep schemes (leapfrog)

  \begin{itemize}
  \tightlist
  \item
    see equation 5
  \item
    want to keep the vector
    \(\left[ \begin{smallmatrix} \hat{u}^{n+1} \\ \hat{u}^n \end{smallmatrix} \right]\)
    to have norm constant or decreasing
  \item
    do this by making sure the matrix is a contracting map (eigenvals
    bounded by 1 in magnitude)
  \end{itemize}
\end{itemize}

\hypertarget{lab-7}{%
\subsection{Lab 7}\label{lab-7}}

stability regions \texttt{:(}

\begin{itemize}
\tightlist
\item
  temporal stability plot:

  \begin{itemize}
  \tightlist
  \item
    form the differential equation
    \(\frac{\partial v}{\partial t} = \gamma \cdot v\)
  \item
    approximate the time derivative using the time deriv approx from the
    scheme
  \item
    find where the solution is stable or decays in time (be constant or
    go to zero, we dont want it to blow up)

    \begin{itemize}
    \tightlist
    \item
      either solve the scheme explicitly, or use the amplification
      factor type argument
    \end{itemize}
  \item
    set \(\gamma = 1\), plot \(\Delta_t\) in the complex plane
  \end{itemize}
\item
  spatial stability plot

  \begin{itemize}
  \tightlist
  \item
    take the spatial discretization that we have in the scheme and build
    a matrix out of it
  \item
    inside of that matrix, set \(\Delta_x, \Delta_y = 1\)
  \item
    determine \(\Delta_t\) from the CFL condition (will have to try
    multiple)
  \item
    compute the eigenvalues for some number of gridpoints \(N\) and plot
    them on top of the temporal stability (dots)
  \item
    if you have multiple time levels on top of the scheme:

    \begin{itemize}
    \tightlist
    \item
      \textbf{good luck}
    \item
      block matrix system?????????
    \end{itemize}
  \end{itemize}
\end{itemize}

\hypertarget{lab-8}{%
\subsection{Lab 8}\label{lab-8}}

how to make a first order system out of a second order equation (example
is wave equation)

\begin{itemize}
\item
  set \[v = u_t, \qquad w = u_x\] \textgreater{} note: this works for
  second order linear, but idk about others
\item
  substitute one of \(v,w\) into the governing equation
\item
  other one is the ``compatibility condition''

  \begin{itemize}
  \tightlist
  \item
    Lax pairs? \(v_x = w_t\) (not important)
  \end{itemize}
\item
  take those two equations and shove them into the linear system
\item
  \[\vec{v}_t = A \vec{v}_x\]
\item
  \[ \vec{v} = \begin{bmatrix} v \\ w \end{bmatrix} \]
\item
  take the eigenvals, eigenvects of \(A\)
\item
  ????
\item
  look at the lab, it explains things
\end{itemize}

\hypertarget{lab-9}{%
\subsection{Lab 9}\label{lab-9}}

block matrices? idk its hard

when you have a 2D system, you get block matrices when you discretize it

\hypertarget{lab-10-lab-11}{%
\subsection{Lab 10, Lab 11}\label{lab-10-lab-11}}

spectral stuff, not on exam!

\hypertarget{homeworks-2021-12-13-1647}{%
\section{Homeworks 2021-12-13 16:47}\label{homeworks-2021-12-13-1647}}

\begin{itemize}
\tightlist
\item
  taylor expand
\item
  temportal as spatial
\item
  drawing stencil
\item
  truncation errors
\item
  modified equation
\end{itemize}

\begin{quote}
\textbf{Note:} ``using Fourier analysis'' or ``taking a Fourier
transform'' just means plug in a Fourier mode
\end{quote}

\hypertarget{well-posedness-hw2}{%
\subsection{Well-posedness (hw2)}\label{well-posedness-hw2}}

3 conditions:

\begin{itemize}
\tightlist
\item
  existence
\item
  uniqueness
\item
  varies continuously with initial condition

  \begin{itemize}
  \tightlist
  \item
    bound on the Fourier mode
  \item
    \[ \Vert u(x,t) \Vert_2 \leq A e^{at} \Vert u(x,0) \Vert_2 \]
  \end{itemize}
\end{itemize}

\hypertarget{idk-what-this-part-is}{%
\subsubsection{idk what this part is}\label{idk-what-this-part-is}}

need to restrict the coefficient of the highest order even derivative -
consider the case when that is zero? - go to the next highest order?

\hypertarget{debugging}{%
\section{Debugging}\label{debugging}}

\begin{itemize}
\tightlist
\item
  CFL:

  \begin{itemize}
  \tightlist
  \item
    funny oscillations (Gibbs phenomena)
  \item
    if you take \(\Delta_t\), it should work (this removes the CFL as a
    problem)

    \begin{itemize}
    \tightlist
    \item
      if setting \(\Delta_t\) to be small fixes the problem, then the
      issue is with the CFL
    \item
      if setting it to be small does \textbf{not} fix the problem, then
      the issue is with the method
    \end{itemize}
  \end{itemize}
\item
  Boundary conditions

  \begin{itemize}
  \tightlist
  \item
    if the solution is seeming to ``leak'' out, then there might be an
    issue with the boundary conditions
  \item
    if it looks fine in the middle, but not on the edges, also check BCs
  \end{itemize}
\item
  Initial conditions:

  \begin{itemize}
  \tightlist
  \item
    evolve a really small amount, see if it works
  \item
    if wrong in early timestep, then probably an IC issue
  \end{itemize}
\item
  check indexing
\end{itemize}

\hypertarget{exam}{%
\section{Exam}\label{exam}}

\hypertarget{cfl}{%
\subsection{CFL:}\label{cfl}}

\begin{itemize}
\tightlist
\item
  ratio of exact wavespeed to numerical wavespeed
\item
  if greater than one, then the method is unstable (usually..,)
\end{itemize}

\hypertarget{lax-richmeyer-equivalence-theorem}{%
\subsection{Lax-Richmeyer equivalence
theorem}\label{lax-richmeyer-equivalence-theorem}}

\begin{itemize}
\item
  stability:

  \begin{itemize}
  \tightlist
  \item
    \[ |Q|^{n+1} \leq K(T_f) \qquad \text{as} \qquad \sup_n,  \quad \lim_{\Delta_t, \Delta_x \to 0} \]
  \end{itemize}
\item
  consistency: \(\lim_{\Delta_t, \Delta_x \to 0}\) of modified equation
  is the actual equation (truncation error goes to zero)
\item
  converges: \(\lim_{\Delta_t, \Delta_x \to 0}\) of the grid function,
  is the actual function (pointwise)
\end{itemize}

\textbf{theorem:} A linear scheme that is both \emph{consistent} and
\emph{stable} converges.

\printbibliography

\end{document}
